\documentclass{bioinfo}
\copyrightyear{2005}
\pubyear{2005}
\usepackage{graphicx}
\hyphenation{Complex-Assembly}

% Useful macros:

\newcommand{\qual}{\texttt{qual}}

\begin{document}
\firstpage{1}

\title[BioPax to SBML qual]{Qualitative translation of relations from BioPax to SBML qual}
\author[B\"uchel \textit{et~al}]{Finja B\"uchel\,$^{1,*}$,
and Andreas Zell\,$^1$\footnote{to whom correspondence should be addressed}}
\address{$^{1}$Department of Cognitive Systems, University of Tuebingen, Sand 1,
72076 T\"ubingen, Germany}


\history{Received on XXXXX; revised on XXXXX; accepted on XXXXX}

\editor{Associate Editor: XXXXXXX}

\maketitle

\begin{abstract}

\section{Motivation:}
The Biological Pathway Exchange Language (BioPax) and the Systems Biology Markup Language (SBML) are the two most popular modeling and data exchange languages in systems biology.
The focus of SBML is quantitative modeling and dynamic simulation of molecular interaction pathways, whereas the BioPax specification concentrates on visualization and qualitative analysis of pathway maps.
BioPax describes reactions and relations. In contrast, reactions are the only type of interaction in core SBML.
With the release of the SBML Qualitative Models extension (\qual), it has recently also become possible to describe relations in SBML.
Before this release, relations could not be translated to SBML or were erroneously converted to reactions. Until now, there exists no BioPax to SBML converter that is fully capable to translate both reactions and relations.
\section{Results:}
Here we present the conversion result of the complete nature Pathway Interaction Database (PID), which includes pathways from BioCarta, Reactome, and from the National Cancer institute.
All PID BioPax Level 2 and Level 3 pathway files have been translated to the SBML format including both reactions and relations by using the new \qual{} extension package.
\section{Availability:}
The complete collection of the PID models is freely available on our homepage ..... (\textbf{TODO Finja: create homepage! and and if necessary licence})
\section{Contact:} \href{finja.buechel@uni-tuebingen.de}{finja.buechel@uni-tuebingen.de}
\end{abstract}

\section{Introduction}
The goal of systems biology is the model-driven and understanding of biological and chemical processes across the scale in detailed levels.
The Biological Pathway Exchange Language (BioPax) and the Systems Biology Markup Language (SBML) are common modeling languages that facilitates the exchange and storage of in-silico models.
The BioPax specification aims at exchanging, visualization, and analyzing such processes on a large scale.
BioPax can be used to image metabolic, signaling, molecular, gene regulatory and genetic interaction networks \citep{Demir2010}.
In contrast, SBML is mostly used for quantitative modeling because it offers the possibility to include kinetic equations and enables the interpretation of the interaction network as an differential equation system \citep{Hucka2003}.
The SBML core specification defines reactions in detail but no other relationships between molecules.
%problem
Before the release of the Qualitative Models specification (\qual, see %\href{http://sbml.org/Community/Wiki/SBML\_Level\_3\_Proposals/Qualitative\_Models}
\href{http://sbml.org/Community/Wiki/SBML_Level_3_Proposals/Qualitative_Models}
{www.sbml.org/Community/Wiki/SBML\_Level\_3\_Proposals/Qualitative\_Models}), it was not possible to define relations or to integrate reactions and relations in one model.
%relevance
Furthermore, it was not possible to easily combine or exchange information between different databases if one database uses the BioPax format and the other one the SBML format.
%literature review
So far, there exist several visualization tools, like Cytoscape or CellDesigner, which can handle both formats by using plugins \citep{Mi2011, Funahashi2007, Smoot2011a, Zinovyev2008}).
\citet*{Ruebenacker2009} suggest a new a bridging format to solve the problem of the conversion from BioPax to SBML.
But until now, there exists only converters from SBML to BioPax like The System Biology Format Converter (see http://www.ebi.ac.uk/compneur-srv/sbml/converters/SBMLtoBioPax.html) but no converter for BioPax to SBML, whose conversion would also include relations.
The need to combine both formats to use the knowledge from a multitude of databases in various application becomes more and more urgent.

% what we did
% necessary to mention why we use PID?
In this paper we present a complete conversion of the nature Pathway Interaction Database (PID, \citet{Schaefer2009}) from BioPax Level 2 and Level 3 format to the SBML format including the \qual{} extension.
The pathway conversion is implemented in Java (\textbf{TODO: Java Trademark}) and uses JSBML \citep{Draeger2011} and PaxTools \citep{Demir2010}.
The translated files are freely available on our homepage: ...\textbf{TODO and add licence}
% what we find out?
% what do the results mean (why are they significant?)
\textbf{TODO: Clemens, Florian und Andreas: Habt ihr hier noch konkrete Ideen, was ich noch einbauen k\"onnte? Fehlt etwas?}


\begin{methods}
\section{Material and Methods}
\subsection{The SBML and the Qualitative Models extension}
%%%%%%%%% SBML
The SBML (Systems Biology Markup Language) core specification is a special XML format to describe quantitative models.
Therefore, several classes are defined describing reactive species and their interaction in a model.
The most important classes are \texttt{species}, \texttt{reaction} and several other elements.
A \texttt{species} element describe each kind of molecule and can be further determined with the aid of MIRIAM annotations \citep{Novere2005}.
The SBML core specification provides no possibility to define other relationships than concrete quantitative reactions. \citep{Hucka2003}

%%%%%%%%% SBML qual
The SBML Qualitative Models extension (\qual) introduces qualitative elements, such as \texttt{qualitativeSpecies}, \texttt{symbol}, and \texttt{transition}, providing the necessary means to describe relationships such as enzyme-enzyme relations, protein-protein interactions, interactions of transcription factors and genes, protein-compound interaction, or links to other pathways.
%Although, technically, those processes are chemical reactions, too, it can often be far more useful to represent them by logical states and transitions.
Instead of concentration levels that are transformed continuously via reactions, \texttt{qualitativeSpecies} have discrete states that are changed using \texttt{transition}s.
These \texttt{transitions} consist of \texttt{input} and \texttt{output} elements and at least one \texttt{functionTerm}.
If, in a qualitative model, a protein A is inhibiting a protein B, this would be represented as a \texttt{transition} with \texttt{input} A and \texttt{output} B.
Furthermore, the input element has a \texttt{sign} attribute describing if the relationship between the input and output elemenst is \texttt{positive}, \texttt{negative}, \texttt{dual}, or \texttt{unknown}.
\subsection{The BioPax specification}
The Biology Pathway Exchange Language (BioPax) is an OWL (Web Ontology Language) dialect which based on the RDF syntax.
There is one super class called \texttt{entity} which extends all other BioPax classes.
One distinguishes between two main classes: \texttt{PhysicalEntity} and \texttt{interaction}.
\texttt{PhysicalEntity} describes molecules, like proteins, complexes, small molecules, DNA, or RNA, whereas \texttt{Interaction} defines reactions and relations between the \texttt{PhysicalEntity}s.
\texttt{Interaction} is split into \texttt{Control} and \texttt{Conversion}
which can be separated into the subclasses \texttt{Catalysis}, \texttt{Modulation}, \texttt{Transport}, \texttt{BiochemicalReaction}, and \texttt{Complex\-Assembly}.

BioPax is released level-wise.
The current level is Level 3.
Level 1 is exclusively able to describe metabolic interactions, whereas in Level 2 additionally supports signaling pathways and molecular interactions.
The difference between Level 2 to Level 3 is the ability of Level 3 to model gene regulatory networks and genetic interactions.
For this purpose the control subclass \texttt{TemplateReactionReglulation} and the conversion subclass \texttt{Degradation} are added.
Furthermore, \texttt{PhysicalEntity} is now able to define \texttt{DNAregion} and \texttt{RNAregion}.
Level 3 is not backwards compatible with Level 2, but Level 2 is backward compatibel with Level 1. \citep{Demir2010}
\subsection{Implementation}
The conversion was programmed with Java, using JSBML \citep{Draeger2011} with the Qualitative Models extension, PaxTools, and the KEGG API (\citep{Kanehisa2006}).
PaxTools was used to read the BioPax files and to manipulate the information content.
This information was extended with MIRIAM identifiers for the following databases: Entrez Gene, Omim, Ensembl, UniProt, ChEBI, DrugBank, Gene Ontology, HGNC, PubChem, 3DMET, NCBI Taxonomy, PDBeChem, GlycomeDB, LipidBank, EC-Numbers (enzyme nomeclature) and various KEGG databases (gene, glycan, reaction, compound, drug, pathway, orthology).
Finally, the transformed information was written with the use of JSBML.

\subsection{Conversion of BioPax to SBML \qual}
%%%%%%%%%%%%%%%%%%%%%%%%%%% FIGURE 1 %%%%%%%%%%%%%%%%%%%%%%%%%%%%%%%%%%%%%%%%%%%%%%%%%%
\begin{figure*}[t!h]
\centering \includegraphics[width=0.95\textwidth]{BioPaxSBMLqual.png}
\caption{Conversion from BioPax Level 2 and Level 3 to SBML with the Qualitative Models extension (\qual).
The green rounded rectangles describe the SBML and SBML \qual{} classes, and the blue ones the BioPax elements.
The distinction between BioPax Level 2 and Level 3 elements is visualized with dashed rectangles.
These dashed rectangles denote Level 3 elements which are not available in Level 2.
All other elements occur in both levels.
The ancestry of both BioPax and SBML elements is drawn as blue arrows for BioPax respectively with green arrows for SBML.
The conversion from BioPax to SBML \qual{} is drawn with black lines.
For some BioPax elements, it depends on the enclosed entities if the BioPax element is translated to a reaction or to a relation.
This translation dependency is visualized with black dashed lines.
A detailed translation description of these elements is shown in Table \ref{Tab:BioPax2SBML}.}
\label{fig:BioPaxSBMLqual}
\end{figure*}

%\begin{figure*}[t!h]
%\centering \includegraphics[width=0.96\textwidth]{BioPax2SBMLqual.png}
%\caption{Mapping from BioPax Level 2 data structures to SBML with the qualitative model extension (qual). The red rounded rectangles and lines describe the BioPax Level 2 elements and their ancestry. SBML qual entities and inheritance lines are colored green. The conversion from BioPax Level 2 to SBML qual is denoted with black lines. For some elements, it depends on the enclosed element entities if BioPax is translate to a reaction or a relation. These are visualized as black dashed lines. The translation of these elements is shown in more detail in Table \ref{Tab:BioPax2SBML}.}\label{fig:BioPax2SBMLqual}
%\end{figure*}
%
%\begin{figure*}[t!h]
%\centering \includegraphics[width=0.96\textwidth]{BioPax3SBMLqual.png}
%\caption{Conversion from BioPax Level 3 to SBML with the qualitative model extension (qual). The blue rounded rectangles and lines describe the BioPax Level 3 elements and how they are inherited. SBML qual entities and inheritance lines are colored green. The conversion from BioPax Level 3 to SBML qual is denoted with black lines. For some elements, it depends on the enclosed element entities if BioPax is translate to a reaction or a relation. These are visualized as black dashed lines. The translation of these elements is shown in more detail in Table \ref{Tab:BioPax2SBML}.}\label{fig:BioPax3SBMLqual}
%\end{figure*}
%%%%%%%%%%%%%%%%%%%%%%%%%%% END FIGURE 1 %%%%%%%%%%%%%%%%%%%%%%%%%%%%%%%%%%%%%%%%%%%%%%
The complete nature Pathway Interaction Database (PID) is converted from BioPax Level 2 and Level 3 to SBML including the Qualitative Models extension.

PID provides NCI-Nature curated pathways, pathways from BioCarta from June
2004, and Reactome human pathways from Version 22, whereas the Reactome data is updated as soon as new pathway informations are available \citep{Schaefer2009}.
The translation of the BioPax pathway files is performed in five steps.
An overview of the mapping from BioPax elements to SBML and to SBML \qual{} elements is shown in Figure \ref{fig:BioPaxSBMLqual}.
%%%%%%%%%%%%%% 1. Step %%%%%% determine organism
In the first step, the pathway organism is determined by searching for the BioSource reference in the BioPax file.
The default organism is human.
%%%%%%%%%%%%%% 2. Step %%%%%% model creation
In the second step, the SBML \texttt{model} and \texttt{qualitativeModel} are build.
The models correspond to the complete pathway represented in the BioPax file.
%%%%%%%%%%%%%% 3. Step %%%%%% entities and annotation
% conversion of Entities, gene id...
In the third step, for each \texttt{PhysicalEntity} a SBML \texttt{species} and \texttt{qualitativeSpecies} is created.
Depending of the kind of the \texttt{PhysicalEntity}, i.e. if it is a protein, complex, DNA, RNA, or small molecule, the species is annotated with the corresponding SBO term \textbf{TODO: Do we need a citation here?}.
Furthermore, the compartment is assigned to the \texttt{species}. If it is not possible to determine the compartment, a default compartment is assigned.
Then, it is searched if there exists an RDF link from the \texttt{PhysicalEntity} to the corresponding Entrez Gene ID, because the Gene ID facilitates the annotation in the fifth step.
If there exists no Gene ID, the gene symbol is search and mapped to a Gene ID.
If neither a gene id nor a gene symbol is available the name of the \texttt{PhysicalEntity} is used.
%%%%%%%%%%%%%% 4. Step %%%%%% reaction relations
In the fourth step, the BioPax interaction elements are converted.
Interaction elements can be split in \texttt{Conversion} and \texttt{Control} elements.
Conversion are mainly converted into SBML reactions except of the \texttt{Transport} and the \texttt{Degradation} elements.
These are translated into transitions.
In BioPax Level 3 the stoichiometry of the reactants and products of \texttt{BiochemicalReaction}s and \texttt{TransportWithBiochemicalReaction}s can also be translated.
Level 2 does not provide stoichiometric information.

The conversion of the \texttt{Control} elements is straight forward, whereas the conversion of \texttt{Control} elements is more sophisticated, because translation into a transition or a reaction depends on the enclosed \texttt{Control} elements.
\texttt{Control} elements always consist of zero or one \texttt{Controller} and zero or more \texttt{Controlled} elements.
\texttt{]Controller} elements can be inherited from \texttt{PhysicalEntity} or \texttt{Pathway}, whereas \texttt{Controlled} elements also are \texttt{Interaction} elements.
Thus, it depends on the kind of \texttt{Controller} and the \texttt{Controlled} element if the \texttt{Interaction} is translated to a SBML reaction or transition.
If the \texttt{Controller} or the \texttt{Controlled} element is a \texttt{Pathway} element the \texttt{Interaction} is always converted to a transition, because in biology it is not possible to create a reaction with a complete pathway as an reactant or a product.
In the other cases, an interaction is translated to a transition if the \texttt{Controlled} element is translated to a transition, too.
For instance, the conversion of a \texttt{Modulation} consisting of a \texttt{PhysicalEntity} as \texttt{Controller} and a \texttt{ComplexAssembly} as \texttt{Controlled} is translated to a reaction.
But if the \texttt{Controlled} element is a degradation the \texttt{Modulation} is converted into a transition.
A detailed overview of the conversion of the \texttt{Control} elements is shown in Table \ref{Tab:BioPax2SBML}.

%%%%%%%%%%%%%%%%%%%%%%%%%%% TABLE 1 %%%%%%%%%%%%%%%%%%%%%%%%%%%%%%%%%%%%%%%%%%%%%%%%%
\begin{table}[t!h]
\processtable{Description of the translation of BioPax control elements\label{Tab:BioPax2SBML}}
{\begin{tabular}{llll}\toprule
BioPax controller & BioPax controlled               & Converted\\
                  &                                 & SBML \qual\\
                  &                                 & element\\
\midrule
\textbf{BioPax Level 3}\\
\midrule
PhysicalEntity & BiochemicalReaction                & reaction\\
PhysicalEntity & ComplexAssembly                    & reaction\\
PhysicalEntity & Control                            & transition\\
PhysicalEntity & Degradation                        & transition\\
PhysicalEntity & Transport                          & transition\\
PhysicalEntity & TransportWithBiochemicalReaction   & reaction\\
PhysicalEntity & Pathway                            & transition\\
PhysicalEntity & TemplateReaction                   & transition\\
\\
Pathway         & BiochemicalReaction               & transition\\
Pathway         & ComplexAssembly                   & transition\\
Pathway         & Control                           & transition\\
Pathway         & Degradation                       & transition\\
Pathway         & Pathway                           & transition\\
Pathway         & TemplateReaction                  & transition\\
Pathway         & Transport                         & transition\\
Pathway         & TransportWithBiochemicalReaction  & transition\\
\\\midrule
\textbf{BioPax Level 2}\\
\midrule
physicalEntity & biochemicalReaction                & reaction\\
physicalEntity & complexAssembly                    & reaction\\
physicalEntity & control                            & transition\\
physicalEntity & pathway                            & transition\\
physicalEntity & transport                          & transition\\
physicalEntity & transportWithBiochemicalReaction   & reaction\\
\\
pathway         & biochemicalReaction               & transition\\
pathway         & complexAssembly                   & transition\\
pathway         & control                           & transition\\
pathway         & pathway                           & transition\\
pathway         & transportWithBiochemicalReaction  & transition\\
pathway         & transport                         & transition\\\botrule
\end{tabular}}{BioPax control elements consists of a controller and one or more controlled elements.
Depending on the kind of controller or controlled element, an entity is translated to a reaction or a transition.
The table gives an overview of this conversion regarding BioPax Level 2 and BioPax Level 3.}
\end{table}
%%%%%%%%%%%%%%%%%%%%%%%%%%% END TABLE 1 %%%%%%%%%%%%%%%%%%%%%%%%%%%%%%%%%%%%%%%%%%%%%%%

%%%%%%%%%%%%%%%% Step 5 %%%%% annotation %%%%%%%%
%% CLEMENS: MIRIAM und SBO                     %%
%%%%%%%%%%%%%%%%%%%%%%%%%%%%%%%%%%%%%%%%%%%%%%%%%
In the fifth and final step, the SBML instances are further annotated.
The BioPax specification allows users to encode arbitrary identifiers for elements.
These can be identifiers for various databases, e.g., UniProt, Entrez Gene, Ensembl, etc. Unfortunately, the syntax used in BioPax is not very consistent which leads to XML elements like TODO-X or TODO-Y within BioPax documents that hamper the automatic reading and interpretation of those models by third party applications.
%%%%%%%%%%% Beispiel %%%%%%%%%%%%%%%%%%%%
\textbf{TODO: Rewrite this example!!!}
  <bp:unificationXref rdf:ID="unificationXref14">
    <bp:DB rdf:datatype="http://www.w3.org/2001/XMLSchema\#string">UniProt</bp:DB>
    <bp:ID rdf:datatype="http://www.w3.org/2001/XMLSchema\#string">Q4KMY3</bp:ID>
  </bp:unificationXref>
  <bp:unificationXref rdf:ID="unificationXref14">
    <bp:DB rdf:datatype="http://www.w3.org/2001/XMLSchema\#string">UniProtKB</bp:DB>
    <bp:ID rdf:datatype="http://www.w3.org/2001/XMLSchema\#string">Q4KMY3</bp:ID>
  </bp:unificationXref>
  <bp:unificationXref rdf:ID="unificationXref14">
    <bp:DB rdf:datatype="http://www.w3.org/2001/XMLSchema\#string">UniProt</bp:DB>
    <bp:ID rdf:datatype="http://www.w3.org/2001/XMLSchema\#string">Q4KMY3, Q7Z5L2</bp:ID>
  </bp:unificationXref>
%%%%%%%%%%% Beispiel ENDE %%%%%%%%%%%%%%%%%%%%

In SBML, such identifiers can be expressed as standardized MIRIAM URNs that can be added as annotation to any SBML element.
We support and add MIRIAM identifiers for the following databases: Entrez Gene, Omim, Ensembl, UniProt, ChEBI, DrugBank, Gene Ontology, HGNC, PubChem, 3DMET, NCBI Taxonomy, PDBeChem, GlycomeDB, LipidBank, EC-Numbers (enzyme nomeclature) and various KEGG databases (gene, glycan, reaction, compound, drug, pathway, orthology).
All supported identifiers from the BioPax files are parsed, some are manually curated (see example of inconsistent usage above) and many annotations are supplemented by additional queries to the KEGG API for every translated element.
The goal of those annotations is to provide models that can be used directly by many researchers, no matter what identifiers or databases they use.

TODO: Typen (protein, gen, etc.) als SBO term in detail auflisten und erklaeren.

\end{methods}


\section{Results and Discussion}
%http://pid.nci.nih.gov/about.shtml#pid link for citation purposes
Warum ist es so toll einen converter von biopax to sbml \qual{} zu haben?
\begin{itemize}
\item we can exchange and combine information from different databases using different model languages
\item TODO: Conversion/Control elemente nochmal zus�tzlich einzeichnen?
\item Eine example conversion einbauen? -> evtl bei results and discussion um zu zeigen, warum nicht alle relationen einwandfrei \"ubersetzt werden k\"onnen
    \item \textbf{TODO: Clemens, Florian und Andreas: Habt ihr hier noch konkrete Erg\"anzungen?}
\end{itemize}

\section{Conclusion}
Conversions between different formats are important in all parts of computer sciences. Many conversions, in general, have errors or come with loss of information. The BioPax to SBML conversion is such an example. Due to limitations of the SBML specification, it was simply not possible to include all information from BioPax files in SBML files, while producing correct SBML code. But with SBML Level 3 and the addition of extensions to the specifications, in particular the \qual{} extension, it is now possible to create accurate and specification conform SBML code, and minimize or even eliminate the loss of information.

BioPax is an RDF format that defines various derived entities that can be genes, proteins, small molecules, etc. These can be translated to SBML species and the type of the entity can be encoded as SBO term or MIRIAM annotation on the species itself. Relations between entities (which correspond to edges in a pathway picture) are also provided with detailed information in BioPax. These can be transports, biochemical reactions, complex assemblies, etc. And this is the point where most conversions to SBML usually produce errors or have a massive loss of information. The SBML core specification only provides reactions, which represent real biochemical reactions with substrates, products and enzymes. But processes like the transport or modulation of an entity can not directly be encoded as a reaction, at least without knowing the exact chemical equation. Hence, former conversions from BioPax to SBML did either convert those relations to incorrect reactions or simply remove them during translation. To fill this gap, the SBML community has very recently released the \qual{} specification, which allows to model arbitrary transitions between species. Using this extension, we produce error-free SBML and minimize or even eliminate the loss of information during the translation.

The SBML models provided with this publication consist of SBML-species and, wherever possible, exact reaction equations. Furthermore all relations from the BioPax documents that could not be converted to exact reactions have been included as qualitative transitions between the species. Additional information, like various identifiers or the type of an entity, are encoded as SBO terms or MIRIAM URNs of the corresponding elements. Furthermore, many information are added beyond the scope of the BioPax document by utilizing the KEGG API.

This results in comprehensive and correct SBML models, created for all pathways in the nature pathway interaction database, that can be downloaded from \href{http://TODO_INSERT_HOMEPAGE_HERE.de}{http://TODOINSERTHOMEPAGEHERE.de}. These models can easily be used, e.g., for further simulation and modeling steps, without having to deal with incorrect input file formats or error-prone conversions.

%TODO (DONE): mehr details von sbml core reations und qual relations und zusammenfassung der uebersetzung, inklusive MIRIAM annotations, sbo terms, etc. Letztendlich auf konvertierte pathways (+URL) als ultimatives ergebnis hinweisen.

\begin{itemize}
\item we provide a conversion from the most important database formats
\item we can exchange and combine information from different databases using different model languages
\item \textbf{TODO: Clemens, Florian und Andreas: Habt ihr hier noch konkrete Erg\"anzungen?}
\end{itemize}


\section*{Acknowledgement}
We thank xy.

\paragraph{Funding\textcolon} German Federal Ministry of Education and Research (BMBF) [National Genome Research Network (NGFN+) under grant number 01GS08134]. \textbf{TODO Virtual liver}

\bibliographystyle{natbib}
%\bibliographystyle{achemnat}
%\bibliographystyle{plainnat}
%\bibliographystyle{abbrv}
%\bibliographystyle{bioinformatics}
%\bibliographystyle{plain}
%
\bibliography{document}

\end{document} 