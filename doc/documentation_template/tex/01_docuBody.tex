% 01_Introduction
\chapter{Introduction}

TODO: Write some text. Write some Introduction for your application.


%%%%%%%%%%%%%%%%%%%%%%%%%%%%%%%%%%
% 02_Installation
%%%%%%%%%%%%%%%%%%%%%%%%%%%%%%%%%%
\chapter{Installation}

TODO: This is some template text. Please modify / replace it with your own.


\section{Requirements}
\subsection{Software}

InCroMAP is entirely written in Java\TTra and runs on any operating system
where a suitable Java Virtual Machine (JDK version 1.6 or newer) is installed.
See, for example, the Java SE download
page\footnote{\url{http://www.oracle.com/technetwork/java/javase/downloads/index.html}\label{fn:jvmldl}}.

\subsection{Hardware}

With at least 1\,GB main memory, you should be able to perform most tasks
without any problem. For large datasets, you should have at least 2\,GB of main
memory. \newline An active internet connection is required for most operations.

\section{Starting the application}
\label{startingTheProgram}

If you downloaded a ZIP-file, you need to unzip it before starting the
application. Depending on your operating system, you should use the provided
shell scripts for starting the application. This is \texttt{start.sh} for Linux
or \texttt{start.bat} for Windows. On MAC OS, you have to create your own
shortcut. You can start the application on all operating systems by typing

\begin{lstlisting}[language=bash,numbers=none]
java -jar -Xms128m -Xmx1024m InCroMAP.jar
\end{lstlisting}

\noindent on your command prompt. Please note that you might have to change
\texttt{InCroMAP.jar} for the real name of the JAR-file, e.g.,
\texttt{InCroMAP1.2.0.jar}. In this example, a minimum of 128\,MB and a maximum
of 1024\,MB of memory will be available for the program. In most cases,
InCroMAP needs more than 128\,MB memory, so it might be convenient to create a
shortcut and start the application with as much memory as available. If you
have 2\,GB RAM, for example, you might want to start the application with the
following command:

\begin{lstlisting}[language=bash,numbers=none]
java -Xms128m -Xmx1400M -jar InCroMAP.jar
\end{lstlisting}

For your convenience, we already created several start-scripts to run the
application with as much memory as possible. How much memory you actually need
strongly depends on the size of your input datasets.
%

\chapter{How to get started}

TODO: Write some text.


% 03_Troubleshooting
\chapter{FAQ / Troubleshooting}
\label{ch:faq}

TODO: These are some template questions. Add new ones and modify/ remove the
old ones to suit your needs.

\noindent \textbf{Where can I get help for a certain component/ option/ checkbox/ etc.?}\newline
Most elements in InCroMAP have tooltips. If you don't understand an option, you
can get help in the first place by just pointing the mouse cursor over it and
wait for the tooltip to show up ($\sim$ 3 seconds).\newline

\noindent \textbf{I'm getting a ``java.lang.OutOfMemoryError: Java heap space"}\newline
Some operations need a lot of memory. If you simply start InCroMAP, without any
JVM parameters, only 64\,MB of memory are available. Please append the argument
\texttt{-Xmx1024M} to start the application with 1\,GB of main memory. See
Section~\vref{startingTheProgram} for a more detailed description of how to
start the application with additional memory. If possible, you should give the
application 2\,GB of main memory. A minimum of 1\,GB main memory should be
available to the application.\newline

\noindent \textbf{Is an internet connection required to run InCroMAP?}\newline
An internet connection is required for most operations. Many identifier mapping
files and pathway-based visualizations require an active internet connection.
However, if you import your data directly with NCBI Entrez Gene IDs and do not
use the pathway-visualization or GO-enrichment, you should be able to run the
application offline.\newline

\noindent \textbf{Where can I get the latest version?}\newline
Go to \url{http://www.cogsys.cs.uni-tuebingen.de/software/Integrator/}.\newline

\noindent \textbf{Which Java version must be installed on my computer to launch
InCroMAP?}\newline InCroMAP requires at least Java 1.6. Please see
\url{http://www.java.com/de/download/} to download the latest Java version.

\noindent \textbf{Why does InCroMAP not start on my Mac with Mac OS prior to
10.6 Update 3?}\newline If you try to launch InCroMAP, but the application does
not start and you receive the following error message on the command-line or
Java console of your Mac, you need to update your Java installation:
\begin{verbatim}
Exception in thread "AWT-EventQueue-0" java.lang.NoClassDefFoundError:
    com/apple/eawt/AboutHandler
    at java.lang.ClassLoader.defineClass1(Native Method)
    at java.lang.ClassLoader.defineClass(ClassLoader.java:703)
    ...
\end{verbatim}
The interface \texttt{com.apple.eawt.AboutHandler} was introduced to Java for
Mac OS X 10.6 Update 3. If you have an earlier version of Mac OS or Java,
please update your OS or Java installation. Also see the Mac OS documentation
about the \texttt{AboutHandler} for more information. On a Mac, you can update
your Java installation through the Software Update menu item in the main Apple
menu.

